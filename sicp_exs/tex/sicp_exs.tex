\documentclass{report}
\usepackage{amsfonts, amsmath, amssymb, hyperref}
\renewcommand{\chaptername}{}
\setlength{\parindent}{0pt}

\title{Solutions to exercises in SICP (2e)}
\author{Paul Tan (\href{http://pauljxtan.com}{website}, \href{https://github.com/pauljxtan}{github})}

\begin{document}

\maketitle
\tableofcontents

\chapter{Exercise 1.10}

This is a pretty fun exercise in rendering Scheme procedures in mathematical notation. Personally, I find it helps with visualizing the recursive process. \\

Ackermann's function is defined in the question as

\begin{equation}
  A(x, y) = \begin{cases}
    0 & \text{if } y = 0 \\
    2y & \text{if } x = 0 \\
    2 & \text{if } y = 1 \\
    A(x - 1, A(x, y - 1)) & \text{otherwise}
  \end{cases}
\end{equation}

For each of the given expressions, we have

\begin{align*}
  A(1, 10) &= A(0, A(1, 9)) \\
           &= A(0, A(0, A(1, 8))) \\
           &= A(0, A(0, A(0, A(1, 7)))) \\
           &= A(0, A(0, A(0, A(0, A(1, 6))))) \\
           &= A(0, A(0, A(0, A(0, A(0, A(1, 5)))))) \\
           &= A(0, A(0, A(0, A(0, A(0, A(0, A(1, 4))))))) \\
           &= A(0, A(0, A(0, A(0, A(0, A(0, A(0, A(1, 3)))))))) \\
           &= A(0, A(0, A(0, A(0, A(0, A(0, A(0, A(0, A(1, 2))))))))) \\
           &= A(0, A(0, A(0, A(0, A(0, A(0, A(0, A(0, A(0, A(1, 1)))))))))) \\
           &= A(0, A(0, A(0, A(0, A(0, A(0, A(0, A(0, A(0, 2))))))))) \\
           &= A(0, A(0, A(0, A(0, A(0, A(0, A(0, A(0, 2\cdot2)))))))) \\
           &= A(0, A(0, A(0, A(0, A(0, A(0, A(0, 2\cdot2\cdot2))))))) \\
           &= A(0, A(0, A(0, A(0, A(0, A(0, 2\cdot2\cdot2\cdot2)))))) \\
           &= A(0, A(0, A(0, A(0, A(0, 2\cdot2\cdot2\cdot2\cdot2))))) \\
           &= A(0, A(0, A(0, A(0, 2\cdot2\cdot2\cdot2\cdot2\cdot2)))) \\
           &= A(0, A(0, A(0, 2\cdot2\cdot2\cdot2\cdot2\cdot2\cdot2))) \\
           &= A(0, A(0, 2\cdot2\cdot2\cdot2\cdot2\cdot2\cdot2\cdot2)) \\
           &= A(0, 2\cdot2\cdot2\cdot2\cdot2\cdot2\cdot2\cdot2\cdot2) \\
           &= 2\cdot2\cdot2\cdot2\cdot2\cdot2\cdot2\cdot2\cdot2\cdot2 \\
           &= 2^{10}
\end{align*}

In general, it appears that for $n > 0$ we have

\begin{equation}
  A(1, n) = 2^n
\end{equation}

(Recall that $A(1, 0) = 0$ by definition.) \\

Next expression:

\begin{align*}
  A(2, 4) &= A(1, A(2, 3)) \\
          &= A(1, A(1, A(2, 2))) \\
          &= A(1, A(1, A(1, A(2, 1)))) \\
          &= A(1, A(1, A(1, 2))) \\
          &= A(1, A(1, 2^2)) \quad &\text{using the result found above} \\
          &= A(1, 2^{2^2}) \quad &\text{ditto} \\
          &= 2^{2^{2^2}} \quad &\text{and again} \\
          &= 2^{2^4} \\
          &= 2^{16} \\
          &= 65536
\end{align*}

And one more:

\begin{align*}
  A(3, 3) &= A(2, A(3, 2)) \\
          &= A(2, A(2, A(3, 1))) \\
          &= A(2, A(2, 2)) \\
          &= A(2, A(1, A(2, 1)) \\
          &= A(2, A(1, 2)) \\
          &= A(2, A(0, A(1, 1))) \\
          &= A(2, A(0, 2)) \\
          &= A(2, 4) \\
          &= A(1, A(2, 3)) \\
          &= A(1, A(1, A(2, 2))) \\
          &= A(1, A(1, A(1, A(2, 1)))) \\
          &= A(1, A(1, A(1, 2))) \\
          &= A(1, A(1, A(0, A(1, 1)))) \\
          &= A(1, A(1, A(0, 2))) \\
          &= A(1, A(1, 4)) & \text{familiar pattern emerging\dots} \\
          &= A(1, A(0, A(1, 3))) \\
          &= A(1, A(0, A(0, A(1, 2)))) \\
          &= A(1, A(0, A(0, A(0, A(1, 1))))) \\
          &= A(1, A(0, A(0, A(0, 2)))) \\
          &= A(1, A(0, A(0, 4))) \\
          &= A(1, A(0, 8)) \\
          &= A(1, 16) \\
          &= 2^{16} \\
          &= 65536 \\
\end{align*}

(Gotta say, the way the equations ``fan'' in and out like that is pretty neat. First it fans out 3 times, then 2, then just once on the final expansion.) \\

Just to check, I computed these values in \textbf{sicp\_1-10\_check.scm} and they are indeed correct. Phew! \\

The last part is essentially the same idea, but with a variable parameter $n$. Let's see\dots

\begin{align*}
  f(n) &= A(0, n) \\
       &= 2n
\end{align*}

\begin{align*}
  g(n) &= A(1, n) \\
       &= A(0, A(1, n - 1)) \\
       &= 2 \cdot A(1, n - 1) \\
       &= 2 \cdot A(0, A(1, n - 2)) \\
       &= 2 \cdot 2 \cdot A(1, n - 2) \\
       &= \dots & \text{follow the pattern\dots} \\
       &= 2^{n - 1} \cdot A(1, 1) \\
       &= 2^{n - 1} \cdot 2 \\
       &= 2^n & \text{as found earlier!}
\end{align*}

\begin{align*}
  h(n) &= A(2, n) \\
       &= A(1, A(2, n - 1)) \\
       &= A(1, A(1, A(2, n - 2))) \\
       &= A(1, A(1, A(1, A(2, n - 3)))) \\
\end{align*}

At this point, note that the levels of nesting is equal to the number $m$ in $n - m$ at the end\dots

\begin{align*}
  \Rightarrow h(n)
       &= A(1, A(1, A(1, \dots A(1, A(2, 1)) \dots ))) \\
       &= A(1, A(1, A(1, \dots A(1, 2) \dots ))) & \text{We know what $A(1, n)$ is from earlier!} \\
       &= A(1, A(1, A(1, \dots 2^2 \dots ))) & \text{Now we "fold" it back in $n - 1$ times\dots} \\
       &= 2^{2^{2^{\dots}}} & \text{with $n$ "2"s} \\
\end{align*}

That is,

\begin{align*}
  A(2, 1) &= 2 \\
  A(2, 2) &= 2^2 = 4 \\
  A(2, 3) &= 2^{2^2} = 16 \\
  A(2, 4) &= 2^{2^{2^2}} = 65536 \quad \text{(as we found earlier)}
\end{align*}

and so on. To be honest, I have no idea how to formulate this as a ``concise mathematical definition''. Any suggestions welcome\dots

\end{document}
